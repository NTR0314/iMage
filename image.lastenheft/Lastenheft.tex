\documentclass[parskip=full]{scrartcl}
\usepackage[utf8]{inputenc} % use utf8 file encoding for TeX sources
\usepackage[T1]{fontenc}    % avoid garbled Unicode text in pdf
\usepackage[german]{babel}  % german hyphenation, quotes, etc
\usepackage{hyperref}       % detailed hyperlink/pdf configuration
\usepackage{tikz}
\usetikzlibrary{shapes,arrows}
\hypersetup{                % ‘texdoc hyperref‘ for options
pdftitle={SWT1: Lastenheft},%
}
\usepackage{csquotes}       % provides \enquote{} macro for "quotes"
\usepackage[nonumberlist]{glossaries}     % provides glossary commands
\usepackage{enumitem}

\makenoidxglossaries

\newglossaryentry{sws}
{
  name=Schlagwortsuche,
  description={Bildsuche die mit Schlagw\"ortern arbeitet. Bilder werden nach Relevanz zur\"uckgegeben}
}

\newglossaryentry{av}
{
	name=Archivbild,
	plural=Archivbilder,
	description={Vorproduzierten Aufnahmen, die meist über Bildagenturen vertrieben und verkauft werden}
}

\title{SWT1: Lastenheft}
\author{Oswald Zink, 2053536}

\begin{document}

\maketitle
\section{Zielbestimmung}
Die Firma Pear Corp. soll durch das Produkt iMage in der Lage sein \glspl{av} zu speichern und zum Verkaufen anbieten zu k\"onnen. Die Bilder sollen dabei verschieden ausgegeben werden k\"onnen(e.g. Dateiformat, Gr\"osse)

\section{Produkteinsatz}
Das Produkt dient der Firma Pear Corp. zur Verwaltung, \"Anderung und Verkauf von Archivbildern.

Zielgruppe: Mitarbeiter der Firma Pear Corp. und Kunden.

Plattform: Betriebssystem mit Java Support.

\section{Funktionale Anforderungen}
\begin{itemize}[nosep]
\item[FA10] Speichern von Daten auf Firmen-internen Server.
\item[FA20] Hinzuf\"ugen und Entfernen von Bildern durch befugte Mitarbeiter.
\item[FA30] Hinzuf\"ugen und Entfernen von Schlagw\"ortern zu Bildern.
\item[FA40] Gew\"ahrleistet kleine Vorschaubilder f\"ur jedes Bild mit Wasserzeichen.
\item[FA50] Nutzer k\"onnen über eine Schlagwortsuche nach Bildern im Archiv suchen.
\item[FA60] Der Nutzer kann zwischen den Dateiformaten „JPG“, „PNG“ und „TIFF“ wählen.
\item[FA70] Der Nutzer kann den Kaufvorgang abbrechen.
\end{itemize}

\section{Produktdaten}
\begin{itemize}[nosep]
\item[PD10] Die Archivbilder sind zu speichern.
\item[PD20] Schlagw\"orter sind zu den Bildern zu speichern.
\item[PD30] Es sind relevante Daten \"uber die Kunden zu speichern.
\end{itemize}

\section{Nichtfunktionale Anforderungen}
\begin{itemize}[nosep]
\item[NF10] Die \gls{sws} dauert nicht l\"anger als 5 Sekunden.
\item[NF20] Die \gls{sws} liefert bis zu 10 nach Relevanz sortierte Bilder zur\"uck.
\item[NF30] Alle Bilder sind in den 3 Dateitypen: JPG, PNG und TIFF bereitgestellt.
\end{itemize}

\section{Systemmodelle}

\subsection{Szenarien}
\subsubsection{Benutzung der Software}
Der Nutzer startet iMage, und findet mit Hilfe der Schlagwortsuche ein Bild. Er klickt auf das Vorschaubild und w\"ahlt im Men\"u das Dateiformat JPG aus. Er entscheidet sich um und bricht den Kaufvorgang ab.
\subsubsection{Schlagwortsuche}
Beim Suchen eines Bildes nutzt der Kunde die Schlagwortsuche. Dort tippt er Schlagworte ein und erh\"alt innerhalb von 5 Sekunden bis zu 10 nach Relevanz sortierte Bilder.
\subsection{Anwendungsfälle}
\subsubsection{Bildkauf}
\begin{tikzpicture}
\draw (3,0) -- (3,10) -- (13,10) -- (13,0) -- (3,0);
\node (image) at (3.75,9.5) {iMage};

\node (B) at (1,9) {Kunde};
\node (A) at (1,8) [minimum size=1cm,circle,draw] {};
\draw (A) -- (1,6.5); 			%koerper
\draw (0.5,7.25) -- (1.5,7.25);	%arme
\draw (1,6.5) -- (0.5,6);		%linker
\draw (1,6.5) -- (1.5,6);
\node (Kunde) at (1,7.25) {};
\node (Suche) at (10,3) [draw, ellipse] {Suche von Bild};

\node (Kauf) at (5.5,7) [draw, ellipse] {Kauf von Bild};
\draw [green] (Kunde) -- (Kauf);

\node (Format) at (10,9) [draw, ellipse, align=left] {Auswahl von Gr\"osse \\ und Dateiformat};
\draw [green, dashed, ->, very thick] (Kauf) -- (Format);
\path (Kauf) -- (Format) node [midway, above, sloped] (TextNode) {<<includes>>};

\draw [green, dashed, ->, very thick] (Kauf) -- (Suche);
\path (Kauf) -- (Suche) node [midway, above, sloped] (TextNode) {<<includes>>};

\node (C) at (1,5) {Mitarbeiter};
\node (D) at (1,4) [minimum size=1cm,circle,draw] {};
\draw (D) -- (1,2.5); 			%koerper
\draw (0.5,3.25) -- (1.5,3.25);	%arme
\draw (1,2.5) -- (0.5,2);		%linker
\draw (1,2.5) -- (1.5,2);
\node (Hund) at (1,3) {};
\node (Hinz) at (7,1) [draw, align=left, ellipse] {Hinzuf\"ugen/Entfernen\\ von Bildern};
\draw [green] (Hund) -- (Hinz);


\end{tikzpicture}


\textbf{Akteure:} Kunde, Mitarbeiter

\textbf{Anwendungsfälle:} Kauf von Bild, Suche von Bild, Auswahl von Gr\"osse und Dateiformat, Hinzuf\"ugen/Entfernen von Bildern

\textbf{Textuelle Beschreibung:}

\begin{itemize}


\item Name: Kauf von Bild   

\item Teilnehmende Akteure: Kunde

\item Eingangsbedingung: Kunde will Archivbild kaufen

\item Ausgangsbedingung: Kunde er\"ahlt Bild.

\item Ereignisfluss: 
\begin{itemize}
\item Kunde \"offnet Image
\item Kunde sucht Bild mittels Schlagwortsuche
\item Kunde w\"ahlt gew\"unschtes Dateiformat aus
\item Kunde \"ahlt kaufen aus
\item Kunde hat die M\"oglichkeit den Kaufvorgang abzubrechen
\item Kunde bezahlt und erh\"ahlt Bild
\end{itemize}

\end{itemize}

\begin{itemize}


\item Name: Entfernen von Bildern

\item Teilnehmende Akteure: Mitarbeiter

\item Eingangsbedingung: Mitarbeiter will Bild l\"oschen

\item Ausgangsbedingung: Bild gel\"oscht

\item Ereignisfluss: 
\begin{itemize}
\item Mitarbeiter \"offnet Image
\item Mitarbeiter sucht Bild mittels Schlagwortsuche
\item Mitarbeiter w\"ahlt l\"oschen aus
\item Bild gel\"oscht.
\end{itemize}

\end{itemize}

\glsaddall
\printnoidxglossaries

\end{document}
